\documentclass[unicode,11pt,a4paper,oneside,numbers=endperiod,openany]{scrartcl}
\usepackage{minted}
\usepackage[fleqn]{amsmath}
\usepackage{amssymb}
\usepackage{xcolor}
\usepackage{caption}


\documentclass[unicode,11pt,a4paper,oneside,numbers=endperiod,openany]{scrartcl}

\renewcommand{\thesubsection}{\arabic{subsection}}

\documentclass[unicode,11pt,a4paper,oneside,numbers=endperiod,openany]{scrartcl}

\renewcommand{\thesubsection}{\arabic{subsection}}

\documentclass[unicode,11pt,a4paper,oneside,numbers=endperiod,openany]{scrartcl}

\renewcommand{\thesubsection}{\arabic{subsection}}

\input{assignment.sty}
\begin{document}


\setassignment
\setduedate{Wednesday, 12 October 2022, 23:59 AM}

\serieheader{Numerical Computing}{2022}{Student: FULL NAME}{Discussed with: FULL NAME}{Solution for Project 1}{}
\newline

\assignmentpolicy
The purpose of this assignment\footnote{This document is originally based on a SIAM book chapter from \textsl{Numerical Computing with Matlab} from  Clever B. Moler.} is to learn the importance of numerical linear algebra algorithms to solve fundamental  linear algebra problems that occur in search engines.



\section*{PageRank Algorithm }

\subsection{Theory [20 points]}

\begin{enumerate}
\item[(a)] \textbf{What are an eigenvector, an eigenvalue and an eigenbasis?}\\
An eigenvector is a nonzero vector that, when involved in a linear transformation gets stretched and scaled but stays on the same span. \\\\
An eigenvalue, denoted with symbol $\lambda$ represents the factor of how much the eigenvector gets stretched along his direction. \\
The eigenvalue can be negative, as it means the resulting vectors inverts its direction after the linear transformation.  \\\\
An eigenbasis is a diagonal matrix that has eigenvectors as its columns, and eigenvalues along its diagonal.\\
Given a matrix $A$ and a linear transformation $T$, 
if there exists a vector $v$ such that $Tv = \lambda v$,
then $v$ is an eigenvector of $A$ and $\lambda$ is an eigenvalue of A.
If there exists a set of vectors $V$ such that $T(V) = \lambda V$, then $V$ is an eigenbasis of $A$.

% An eigenvector is a vector that doesn't change direction when a linear transformation is applied to it.
% An eigenvalue is a scalar that is multiplied by the eigenvector.
% An eigenbasis is a set of eigenvectors that are linearly independent.

% Given a matrix $A$ and a linear transformation $T$, 
% if there exists a vector $v$ such that $Tv = \lambda v$,
% then $v$ is an eigenvector of $A$ and $\lambda$ is an eigenvalue of A.
% If there exists a set of vectors $V$ such that $T(V) = \lambda V$, then $V$ is an eigenbasis of $A$.


\item[(b)] \textbf{What assumptions should be made to guarantee convergence of the power method?}\\
We need to assume that the eigenvalue to which the power method converges 
is the dominant eigenvalue, and we also need to assume that the randomly-chosen initial vector has a component in the same direction as the eigenvector.\\
Also, to guarantee a faster convergence, $\lambda _1$ and $\lambda _2$ have to be distant, as the asymptotic error constant is $|\frac{\lambda _1}{\lambda _2}|$ meaning convergence is extremely slow if the two eigenvalues are close to each other.

\item[(c)] \textbf{What is the shift and invert approach?}\\
As mentioned previously, if the eigenvalues are close to each other convergence can be extremely slow.\\
The inverse iteration uses the shift and invert tecnique to speed up the convergence.

If the eigenvalues of $A$ are $\lambda_j$, the eigenvalues of $A -\alpha I$ are $\lambda_j - \alpha$, 
and the eigenvalues of $B = (A-\alpha I)^{-1}$ are $\mu_j = \frac{1}{\lambda_j - \alpha}$

If we apply the power method to $B$ we get an improve rate of $|\frac{\mu_2}{\mu_1}| = |\frac{\frac{1}{\lambda_2 - \alpha}}{\frac{1}{\lambda_1 - \alpha}}| = |\frac{\lambda_1 - \alpha}{\lambda_2 - \alpha}|$.\\
It is therefore in our best interests to find an $\alpha$ as close as possible to any eigenvalue.\\
The shift and invert approach is to choose $\alpha$ such that the improve rate is as small as possible.


\item[(d)] \textbf{What is the difference in cost of a single iteration of the power method, compared to the inverse iteration?}\\
Using the power method, each iteration involves a \textit{matrix-vector multiplication}, while the \textit{inverse iteration} requires solving a linear system.
Solving a linear system is way more computationally expensive than a matrix-vector multiplication, having a fast convergence for inverse iteration is necessary in order for it to to be effective; 
this would be impossible when tackling larger issues such as Internet searching.
Inverse iteration must be used for smaller problems or problems with special structure that enables fast direct methods.


\item[(e)] \textbf{What is a Rayleigh quotient and how can it be used for eigenvalue computations?}\\
The Rayleigh quotient is an improvement over the inverse iteration method for finding eigenvalues.\\
It guarantees very fast convergence, in most cases even cubically. To guarantee fast convergence, we change the value of $\alpha$ dynamically to be the Rayleigh quotient in each iteration.\\
Even though Rayleigh quotient iteration is more computationally expensive as it requires to refactor the matrix in each iteration, the cubic convergence makes it worthwhile.

\end{enumerate}

\subsection{Other webgraphs [10 points]}

\subsection{Connectivity matrix and subcliques [5 points]}

\subsection{Connectivity matrix and disjoint subgraphs [10 points]}

\subsection{PageRanks by solving a sparse linear system [40 points]}

\subsection{Quality of the Report [15 points]}


\end{document}

\begin{document}


\setassignment
\setduedate{Wednesday, 12 October 2022, 23:59 AM}

\serieheader{Numerical Computing}{2022}{Student: FULL NAME}{Discussed with: FULL NAME}{Solution for Project 1}{}
\newline

\assignmentpolicy
The purpose of this assignment\footnote{This document is originally based on a SIAM book chapter from \textsl{Numerical Computing with Matlab} from  Clever B. Moler.} is to learn the importance of numerical linear algebra algorithms to solve fundamental  linear algebra problems that occur in search engines.



\section*{PageRank Algorithm }

\subsection{Theory [20 points]}

\begin{enumerate}
\item[(a)] \textbf{What are an eigenvector, an eigenvalue and an eigenbasis?}\\
An eigenvector is a nonzero vector that, when involved in a linear transformation gets stretched and scaled but stays on the same span. \\\\
An eigenvalue, denoted with symbol $\lambda$ represents the factor of how much the eigenvector gets stretched along his direction. \\
The eigenvalue can be negative, as it means the resulting vectors inverts its direction after the linear transformation.  \\\\
An eigenbasis is a diagonal matrix that has eigenvectors as its columns, and eigenvalues along its diagonal.\\
Given a matrix $A$ and a linear transformation $T$, 
if there exists a vector $v$ such that $Tv = \lambda v$,
then $v$ is an eigenvector of $A$ and $\lambda$ is an eigenvalue of A.
If there exists a set of vectors $V$ such that $T(V) = \lambda V$, then $V$ is an eigenbasis of $A$.

% An eigenvector is a vector that doesn't change direction when a linear transformation is applied to it.
% An eigenvalue is a scalar that is multiplied by the eigenvector.
% An eigenbasis is a set of eigenvectors that are linearly independent.

% Given a matrix $A$ and a linear transformation $T$, 
% if there exists a vector $v$ such that $Tv = \lambda v$,
% then $v$ is an eigenvector of $A$ and $\lambda$ is an eigenvalue of A.
% If there exists a set of vectors $V$ such that $T(V) = \lambda V$, then $V$ is an eigenbasis of $A$.


\item[(b)] \textbf{What assumptions should be made to guarantee convergence of the power method?}\\
We need to assume that the eigenvalue to which the power method converges 
is the dominant eigenvalue, and we also need to assume that the randomly-chosen initial vector has a component in the same direction as the eigenvector.\\
Also, to guarantee a faster convergence, $\lambda _1$ and $\lambda _2$ have to be distant, as the asymptotic error constant is $|\frac{\lambda _1}{\lambda _2}|$ meaning convergence is extremely slow if the two eigenvalues are close to each other.

\item[(c)] \textbf{What is the shift and invert approach?}\\
As mentioned previously, if the eigenvalues are close to each other convergence can be extremely slow.\\
The inverse iteration uses the shift and invert tecnique to speed up the convergence.

If the eigenvalues of $A$ are $\lambda_j$, the eigenvalues of $A -\alpha I$ are $\lambda_j - \alpha$, 
and the eigenvalues of $B = (A-\alpha I)^{-1}$ are $\mu_j = \frac{1}{\lambda_j - \alpha}$

If we apply the power method to $B$ we get an improve rate of $|\frac{\mu_2}{\mu_1}| = |\frac{\frac{1}{\lambda_2 - \alpha}}{\frac{1}{\lambda_1 - \alpha}}| = |\frac{\lambda_1 - \alpha}{\lambda_2 - \alpha}|$.\\
It is therefore in our best interests to find an $\alpha$ as close as possible to any eigenvalue.\\
The shift and invert approach is to choose $\alpha$ such that the improve rate is as small as possible.


\item[(d)] \textbf{What is the difference in cost of a single iteration of the power method, compared to the inverse iteration?}\\
Using the power method, each iteration involves a \textit{matrix-vector multiplication}, while the \textit{inverse iteration} requires solving a linear system.
Solving a linear system is way more computationally expensive than a matrix-vector multiplication, having a fast convergence for inverse iteration is necessary in order for it to to be effective; 
this would be impossible when tackling larger issues such as Internet searching.
Inverse iteration must be used for smaller problems or problems with special structure that enables fast direct methods.


\item[(e)] \textbf{What is a Rayleigh quotient and how can it be used for eigenvalue computations?}\\
The Rayleigh quotient is an improvement over the inverse iteration method for finding eigenvalues.\\
It guarantees very fast convergence, in most cases even cubically. To guarantee fast convergence, we change the value of $\alpha$ dynamically to be the Rayleigh quotient in each iteration.\\
Even though Rayleigh quotient iteration is more computationally expensive as it requires to refactor the matrix in each iteration, the cubic convergence makes it worthwhile.

\end{enumerate}

\subsection{Other webgraphs [10 points]}

\subsection{Connectivity matrix and subcliques [5 points]}

\subsection{Connectivity matrix and disjoint subgraphs [10 points]}

\subsection{PageRanks by solving a sparse linear system [40 points]}

\subsection{Quality of the Report [15 points]}


\end{document}

\begin{document}


\setassignment
\setduedate{Wednesday, 12 October 2022, 23:59 AM}

\serieheader{Numerical Computing}{2022}{Student: FULL NAME}{Discussed with: FULL NAME}{Solution for Project 1}{}
\newline

\assignmentpolicy
The purpose of this assignment\footnote{This document is originally based on a SIAM book chapter from \textsl{Numerical Computing with Matlab} from  Clever B. Moler.} is to learn the importance of numerical linear algebra algorithms to solve fundamental  linear algebra problems that occur in search engines.



\section*{PageRank Algorithm }

\subsection{Theory [20 points]}

\begin{enumerate}
\item[(a)] \textbf{What are an eigenvector, an eigenvalue and an eigenbasis?}\\
An eigenvector is a nonzero vector that, when involved in a linear transformation gets stretched and scaled but stays on the same span. \\\\
An eigenvalue, denoted with symbol $\lambda$ represents the factor of how much the eigenvector gets stretched along his direction. \\
The eigenvalue can be negative, as it means the resulting vectors inverts its direction after the linear transformation.  \\\\
An eigenbasis is a diagonal matrix that has eigenvectors as its columns, and eigenvalues along its diagonal.\\
Given a matrix $A$ and a linear transformation $T$, 
if there exists a vector $v$ such that $Tv = \lambda v$,
then $v$ is an eigenvector of $A$ and $\lambda$ is an eigenvalue of A.
If there exists a set of vectors $V$ such that $T(V) = \lambda V$, then $V$ is an eigenbasis of $A$.

% An eigenvector is a vector that doesn't change direction when a linear transformation is applied to it.
% An eigenvalue is a scalar that is multiplied by the eigenvector.
% An eigenbasis is a set of eigenvectors that are linearly independent.

% Given a matrix $A$ and a linear transformation $T$, 
% if there exists a vector $v$ such that $Tv = \lambda v$,
% then $v$ is an eigenvector of $A$ and $\lambda$ is an eigenvalue of A.
% If there exists a set of vectors $V$ such that $T(V) = \lambda V$, then $V$ is an eigenbasis of $A$.


\item[(b)] \textbf{What assumptions should be made to guarantee convergence of the power method?}\\
We need to assume that the eigenvalue to which the power method converges 
is the dominant eigenvalue, and we also need to assume that the randomly-chosen initial vector has a component in the same direction as the eigenvector.\\
Also, to guarantee a faster convergence, $\lambda _1$ and $\lambda _2$ have to be distant, as the asymptotic error constant is $|\frac{\lambda _1}{\lambda _2}|$ meaning convergence is extremely slow if the two eigenvalues are close to each other.

\item[(c)] \textbf{What is the shift and invert approach?}\\
As mentioned previously, if the eigenvalues are close to each other convergence can be extremely slow.\\
The inverse iteration uses the shift and invert tecnique to speed up the convergence.

If the eigenvalues of $A$ are $\lambda_j$, the eigenvalues of $A -\alpha I$ are $\lambda_j - \alpha$, 
and the eigenvalues of $B = (A-\alpha I)^{-1}$ are $\mu_j = \frac{1}{\lambda_j - \alpha}$

If we apply the power method to $B$ we get an improve rate of $|\frac{\mu_2}{\mu_1}| = |\frac{\frac{1}{\lambda_2 - \alpha}}{\frac{1}{\lambda_1 - \alpha}}| = |\frac{\lambda_1 - \alpha}{\lambda_2 - \alpha}|$.\\
It is therefore in our best interests to find an $\alpha$ as close as possible to any eigenvalue.\\
The shift and invert approach is to choose $\alpha$ such that the improve rate is as small as possible.


\item[(d)] \textbf{What is the difference in cost of a single iteration of the power method, compared to the inverse iteration?}\\
Using the power method, each iteration involves a \textit{matrix-vector multiplication}, while the \textit{inverse iteration} requires solving a linear system.
Solving a linear system is way more computationally expensive than a matrix-vector multiplication, having a fast convergence for inverse iteration is necessary in order for it to to be effective; 
this would be impossible when tackling larger issues such as Internet searching.
Inverse iteration must be used for smaller problems or problems with special structure that enables fast direct methods.


\item[(e)] \textbf{What is a Rayleigh quotient and how can it be used for eigenvalue computations?}\\
The Rayleigh quotient is an improvement over the inverse iteration method for finding eigenvalues.\\
It guarantees very fast convergence, in most cases even cubically. To guarantee fast convergence, we change the value of $\alpha$ dynamically to be the Rayleigh quotient in each iteration.\\
Even though Rayleigh quotient iteration is more computationally expensive as it requires to refactor the matrix in each iteration, the cubic convergence makes it worthwhile.

\end{enumerate}

\subsection{Other webgraphs [10 points]}

\subsection{Connectivity matrix and subcliques [5 points]}

\subsection{Connectivity matrix and disjoint subgraphs [10 points]}

\subsection{PageRanks by solving a sparse linear system [40 points]}

\subsection{Quality of the Report [15 points]}


\end{document}


\newcommand{\bmat}[1]{
   \ensuremath{
   \begin{bmatrix}
       #1
   \end{bmatrix}
}}


\begin{document}


\setassignment
\setduedate{Wednesday, November 9, 2022, 11:59 PM}

\serieheader{Numerical Computing}{2022}{Student: Albert Cerfeda}{Discussed with: Alessandro Gobbetti}{Solution for Project 3}{}
\newline

\assignmentpolicy

\tableofcontents

\clearpage
%%%%%%%%%%%%%%%%%%%%%%%%%%%%%%%%%%%%%%%%%%%%%%%%%%
\section{The assignment}
%%%%%%%%%%%%%%%%%%%%%%%%%%%%%%%%%%%%%%%%%%%%%%%%%%

\subsection{Implement various graph partitioning algorithms [50 points]}
For an unweighted graph $G = (V, E)$ its adjacency matrix $A^{n\times n}$ where $n = |V|$ is defined as follows:
\[
A_{i j}=\left\{\begin{array}{ll}
1 & \text { if }(i j) \in E \\
0 & \text { otherwise }
\end{array},\right.
\]
such that the Adjacency matrix contains a $1$ when there is an edge between two vertices and $0$ otherwise. For undirected graphs, the adjacency matrix is symmetrical.\\
\[
\mathbf{A} =\left[\begin{array}{ccc}
0 & 1 & 1 \\
1 & 0 & 1 \\
1 & 1 & 0
\end{array}\right] \text{Adjacency matrix for a fully connected undirected graph where  } |V|=3
\]

\subsubsection*{Spectral Graph Bisection}
The Spectral Graph bisection involves partitioning the \textbf{Fielder vector} (i.e the eigenvector associated with the second smallest eigenvalue) of the graph \textbf{Laplacian matrix} $L\in  \mathbb{R}^{n\times n}$ around some value $m$.\\
The \textbf{graph Laplacian} is a symmetric positive semi-definite matrix that expresses various graph properties. It is computed through the adjacency matrix $A$ and weight matrix $D$.\\
Given a weight matrix $D$:

\[
\mathbf{D}:=\sum_{j=1}^n A_{i j}=\left[\begin{array}{ccc}
2 & 0 & 0\\
0 & 2 & 0\\
0 & 0 & 2
\end{array}\right] \text{such that $D_{jj}$ contains the total degree of vertex $V_j$ }
\]
we define the \textbf{graph Laplacian matrix} as follows:
\[
\mathbf{L}:=\mathbf{D}-\mathbf{A}=\left[\begin{array}{ccc}
2 & -1 & -1\\
-1 & 2 & -1\\
-1 & -1 & 2
\end{array}\right]
\]

The threshold value $m$ around which we define the two partitions can be chosen in two ways:
\begin{enumerate}
\item \textbf{median value} of the chosen eigenvector\\
The two partitions will have homogeneous size.
\item \textbf{$m = 0$}\\
Guarantees a lower amount of edge cuts between nodes.
\end{enumerate}

\begin{minted}{julia}
function spectral_part(A)
    n = size(A)[1]
    if n > 4*10^4
        @warn "graph is large. Computing eigen values may take too long."     
    end

    D = Diagonal(vec(sum(A,dims=1)))
    L = Matrix(D-A);

    e = eigen(L);
    w = e.vectors[:,sortperm(e.values)[2]];

    # m = median(w);
    m = 0
    return map(x->x < m ? 1 : 2, w);
end
\end{minted}
Notice how the chosen threshold value is $0$. This yields fewer edge cuts between the nodes of the two partitions.

\subsubsection*{Inertial Graph Bisection}
Inertial bisection partitions the vertices based on their geometrical location, expressed with a coordinate tuple $(x_i,y_i)$.\\
We trace a line $l$ across the space and partition the vertices based on whether they are on one side of the line or the other.\\
The sum of distances of the vertices from the line $l$ is defined as follows:
\begin{align*}
\sum_{i=1}^n d_i^2 &=\mathbf{u}^T\left[\begin{array}{ll}S_{x x} & S_{x y} \\ S_{x y} & S_{y y}\end{array}\right] \mathbf{u}\\
&=\mathbf{u}^T \mathbf{M} \mathbf{u}
\end{align*}
Matrix $M$ expresses the distance of the vertices from the \textbf{center of mass}, defined as follows:
\[
\bar{x}=\frac{1}{n} \sum_{i=1}^n x_i, \quad \bar{y}=\frac{1}{n} \sum_{i=1}^n y_i
\]
In order to minimize the sum of distances as much as possible we choose $u$ to be a vector orthogonal to the smallest eigenvector for matrix $M$.


\begin{minted}{julia}
function inertial_part(A, coords)
    tuples = eachrow(coords)
    
    cm = reduce((acc,coord)->(acc[1]+coord[1], acc[2]+coord[2]),tuples, init=(0,0));
    cm = (cm[1]/length(tuples),cm[2]/length(tuples))

    sxx = reduce((sxx,coord)->sxx+(coord[1]-cm[1])^2,tuples,init=0)
    syy = reduce((syy,coord)->syy+(coord[2]-cm[2])^2,tuples,init=0) 
    sxy = reduce((sxy,coord)->sxy+(coord[1]-cm[1])*(coord[2]-cm[2]),tuples,init=0)
    M = [ sxx sxy ; sxy syy]

    e, v = eigs(M,nev=1,which=:SR)
    v = v[:,1] # Smallest eigenvector
    w = [ v[2],-v[1] ];

    p = partition(coords,w)
    return map(x->x∈p[1] ? 1 : 2, collect(1:size(A)[1]))

end
\end{minted}

\clearpage
Running the bisective benchmark yields the following results:
\begin{table}[h!]
\caption{$\texttt {bench\_bisection.jl}$ results}
\centering
\begin{tabular}{l|r|r|r|r} \hline\hline 
                Mesh &  Coordinate &    Metis &  Spectral &  Inertial \\
                     &             &  v.5.1.0 &           &          \\
\hline grid(12, 100) &        12.0 &     12.0 &      12.0 &      12.0 \\
       grid(100, 12) &        12.0 &     14.0 &      12.0 &      12.0 \\
 grid(100, 12, -π/4) &        22.0 &     14.0 &      12.0 &      12.0 \\
           gridt(50) &        73.0 &     80.0 &      66.0 &      72.0 \\
           gridt(40) &        59.0 &     62.0 &      52.0 &      59.0 \\
           smallmesh &        25.0 &     13.0 &      12.0 &      30.0 \\
               tapir &        55.0 &     24.0 &      18.0 &      49.0 \\
            eppstein &        42.0 &     40.0 &      42.0 &      45.0 \\
\hline \hline
\end{tabular}
\label{table:bisection}
\end{table}\\
We notice how $\texttt {Coordinate}$ and $\texttt{Metis}$ are the less accurate. The randomness of $\texttt Metis$ has been taken in account of and after running the benchmark multiple times, the average results for $\texttt Metis$ are in the likeliness of the ones reported in the table.\\
$\texttt Spectral$ although being the most accurate is also the slowest as computing the eigenvectors and eigenvalues of large matrices such as matrix $M$ is computationally very expensive.
\begin{figure}[h!]
    \begin{minipage}{0.5\textwidth}
        \centering
        \includegraphics[height=5cm]{fig/plot/bisection/bisection-tapir-metis-cut_80.0}
        \caption{$\texttt {Metis}$ algorithm. \textbf{80 edge cuts}}
    \end{minipage}
        \begin{minipage}{0.5\textwidth}
        \centering
        \includegraphics[height=5cm]{fig/plot/bisection/bisection-tapir-inertial-cut_73.0}
        \caption{$\texttt {Inertial}$ algorithm. \textbf{73 edge cuts}}
    \end{minipage}
    \caption*{Partition comparison for mesh $\texttt {tapir}$}
\end{figure}\\
\begin{figure}[h!]
    \begin{minipage}{0.5\textwidth}
        \centering
        \includegraphics[height=5cm]{fig/plot/bisection/bisection-smallmesh-coordinate-cut_22.0}
        \caption{$\texttt {Coordinate}$ algorithm. \textbf{22 edge cuts}}
    \end{minipage}
        \begin{minipage}{0.5\textwidth}
        \centering
        \includegraphics[height=5cm]{fig/plot/bisection/bisection-smallmesh-spectral-cut_12.0}
        \caption{$\texttt {Spectral}$ algorithm. \textbf{12 edge cuts}}
    \end{minipage}
    \caption*{Partition comparison for mesh $\texttt {smallmesh}$}
\end{figure}\\

\clearpage
\subsection{Recursively bisecting meshes [20 points]}
By running the partitioning algorithms recursively, we split each graph/subgraph $G$ into two more subgraphs $G^'$ and $G^{''}$. Using recursion can pay to our advantage if we intend to parallelize and scale our partition computation. \\
We therefore split the graphs into $2^n$ subgraphs where $n$ is the number of recursion levels.\\
One fundamental difference though is that the two partitions that each recursion level yields \textbf{influence all the next recursions}. For $n=3$ and $n=4$ we split the graph into $4$ and $8$ subgraphs respectively.\\
By running the benchmark we notice how the $\texttt {Spectral}$ really stands out for being the affected the most performance-wise from computing the eigenvectors and eigenvalues of very large graphs.\\ Generally, the recursive approach results in greater edge cuts. $\texttt {Inertial}$ yields low edge cuts and is very fast as it always computes eigenvectors of matrix $M\in \mathbb{R}^{2\times 2}$. 

\begin{figure}[h!]
    \begin{minipage}{0.5\textwidth}
        \centering
        \includegraphics[height=5cm]{fig/plot/recursive/recursive-crack-coordinate-p16-cut_1861.0}
        \caption{$\texttt {Coordinate}$ algorithm.\textbf{1861 edge cuts}}
    \end{minipage}
    \begin{minipage}{0.5\textwidth}
        \centering
        \includegraphics[height=5cm]{fig/plot/recursive/recursive-crack-inertial-p16-cut_1618.0}
        \caption{$\texttt {Inertial}$ algorithm. \textbf{1618 edge cuts}}
    \end{minipage}
        \begin{minipage}{0.5\textwidth}
        \centering
        \includegraphics[height=5cm]{fig/plot/recursive/recursive-crack-spectral-p16-cut_1303.0}
        \caption{$\texttt {Spectral}$ algorithm. \textbf{1303 edge cuts}}
    \end{minipage}
    \begin{minipage}{0.5\textwidth}
        \centering
        \includegraphics[height=5cm]{fig/plot/recursive/recursive-crack-metis-p16-cut_1276.0}
        \caption{$\texttt {Metis}$ algorithm. \textbf{1276 edge cuts}}
    \end{minipage}
    \caption*{Partition comparison for mesh $\texttt {crack}$ with $n=4$ (16 partitioned subgraphs)}
\end{figure}


\begin{table}[h!]
\caption{Edge-cut results for recursive bi-partitioning.}
\begin{tabular}{l|r|r|r|r|r|r|r|r} \hline\hline 
          Mesh & Spectral & Spectral &   Metis &    Metis & Coordinate & Coordinate & Inertial & Inertial \\
               &  8 parts & 16 parts & 8 parts & 16 parts &    8 parts &   16 parts &  8 parts & 16 parts \\
\hline
      airfoil1 &    327.0 &    578.0 &   311.0 &    580.0 &      516.0 &      819.0 &    578.0 &    904.0 \\
 netz4504\_dual &    105.0 &    174.0 &   100.0 &    154.0 &      127.0 &      198.0 &    122.0 &    202.0 \\
         stufe &    124.0 &    216.0 &   112.0 &    197.0 &      123.0 &      228.0 &    135.0 &    268.0 \\
          3elt &    372.0 &    671.0 &   418.0 &    651.0 &      733.0 &     1168.0 &    880.0 &   1342.0 \\
        barth4 &    505.0 &    758.0 &   491.0 &    773.0 &      875.0 &     1306.0 &    892.0 &   1350.0 \\
       ukerbe1 &    118.0 &    224.0 &   125.0 &    241.0 &      225.0 &      374.0 &    280.0 &    469.0 \\
         crack &    804.0 &   1303.0 &   759.0 &   1276.0 &     1344.0 &     1861.0 &   1061.0 &   1618.0 \\


 \hline \hline
\end{tabular}
\label{table:Rec_bisection}
\end{table}

\clearpage
\subsection{Comparing recursive bisection to direct $k$-way partitioning [15 points]}
Running the $\texttt bench_recursive.jl$ yields the following results.\\
In our case we notice that our count of edge cuts is almost double in larger graphs when partitioning in $32$ subgraphs, either with $k\texttt {-way}$ or recursive partitioning.
\begin{figure}[h!]
    \begin{minipage}{0.5\textwidth}
        \centering
        \includegraphics[height=5cm]{fig/plot/metis/metis-commanche_dual-kway-p16-cut_336.0.png}
        \caption{$k$-way partitioning.\\\textbf{336 edge cuts}}
    \end{minipage}
    \begin{minipage}{0.5\textwidth}
        \centering
        \includegraphics[height=5cm]{fig/plot/metis/metis-commanche_dual-recursive-p16-cut_377.0.png}
        \caption{Recursive partitioning.\\\textbf{377 edge cuts}}
    \end{minipage}
    \caption*{Partitioning comparison for mesh $\texttt commanche\_dual$ with $n=4$ (16 partitioned subgraphs)}
\end{figure}
\begin{figure}[h!]
    \begin{minipage}{0.5\textwidth}
        \centering
        \includegraphics[height=5cm]{fig/plot/metis/metis-skirt-kway-p16-cut_3307.0.png}
        \caption{$k$-way partitioning .\\\textbf{3307 edge cuts}}
    \end{minipage}
    \begin{minipage}{0.5\textwidth}
        \centering
        \includegraphics[height=5cm]{fig/plot/metis/metis-skirt-recursive-p16-cut_6364.0.png}
        \caption{Recursive partitioning.\\\textbf{3262 edge cuts}}
    \end{minipage}
    \caption*{Partitioning comparison for mesh $\texttt skirt$ with $n=4$ (16 partitioned subgraphs)}
\end{figure}

\begin{table}[h!]
\centering
\begin{tabular}{l|r|r} \hline\hline 
              Partitions & Helicopter &  Skirt \\
\hline
  16-recursive bisection &      377.0 & 3262.0 \\
 16-way direct bisection &      336.0 & 3307.0 \\
  32-recursive bisection &      578.0 & 6364.0 \\
 32-way direct bisection &      528.0 & 6095.0 \\
 \hline \hline
\end{tabular}              
\label{table:Compare_Metis}
\end{table}

\end{document}
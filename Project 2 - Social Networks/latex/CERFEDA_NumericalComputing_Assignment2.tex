\documentclass[unicode,11pt,a4paper,oneside,numbers=endperiod,openany]{scrartcl}

\renewcommand{\thesubsection}{\arabic{subsection}}
\usepackage{amsmath}


\documentclass[unicode,11pt,a4paper,oneside,numbers=endperiod,openany]{scrartcl}

\renewcommand{\thesubsection}{\arabic{subsection}}

\documentclass[unicode,11pt,a4paper,oneside,numbers=endperiod,openany]{scrartcl}

\renewcommand{\thesubsection}{\arabic{subsection}}

\documentclass[unicode,11pt,a4paper,oneside,numbers=endperiod,openany]{scrartcl}

\renewcommand{\thesubsection}{\arabic{subsection}}

\input{assignment.sty}
\begin{document}


\setassignment
\setduedate{Wednesday, 12 October 2022, 23:59 AM}

\serieheader{Numerical Computing}{2022}{Student: FULL NAME}{Discussed with: FULL NAME}{Solution for Project 1}{}
\newline

\assignmentpolicy
The purpose of this assignment\footnote{This document is originally based on a SIAM book chapter from \textsl{Numerical Computing with Matlab} from  Clever B. Moler.} is to learn the importance of numerical linear algebra algorithms to solve fundamental  linear algebra problems that occur in search engines.



\section*{PageRank Algorithm }

\subsection{Theory [20 points]}

\begin{enumerate}
\item[(a)] \textbf{What are an eigenvector, an eigenvalue and an eigenbasis?}\\
An eigenvector is a nonzero vector that, when involved in a linear transformation gets stretched and scaled but stays on the same span. \\\\
An eigenvalue, denoted with symbol $\lambda$ represents the factor of how much the eigenvector gets stretched along his direction. \\
The eigenvalue can be negative, as it means the resulting vectors inverts its direction after the linear transformation.  \\\\
An eigenbasis is a diagonal matrix that has eigenvectors as its columns, and eigenvalues along its diagonal.\\
Given a matrix $A$ and a linear transformation $T$, 
if there exists a vector $v$ such that $Tv = \lambda v$,
then $v$ is an eigenvector of $A$ and $\lambda$ is an eigenvalue of A.
If there exists a set of vectors $V$ such that $T(V) = \lambda V$, then $V$ is an eigenbasis of $A$.

% An eigenvector is a vector that doesn't change direction when a linear transformation is applied to it.
% An eigenvalue is a scalar that is multiplied by the eigenvector.
% An eigenbasis is a set of eigenvectors that are linearly independent.

% Given a matrix $A$ and a linear transformation $T$, 
% if there exists a vector $v$ such that $Tv = \lambda v$,
% then $v$ is an eigenvector of $A$ and $\lambda$ is an eigenvalue of A.
% If there exists a set of vectors $V$ such that $T(V) = \lambda V$, then $V$ is an eigenbasis of $A$.


\item[(b)] \textbf{What assumptions should be made to guarantee convergence of the power method?}\\
We need to assume that the eigenvalue to which the power method converges 
is the dominant eigenvalue, and we also need to assume that the randomly-chosen initial vector has a component in the same direction as the eigenvector.\\
Also, to guarantee a faster convergence, $\lambda _1$ and $\lambda _2$ have to be distant, as the asymptotic error constant is $|\frac{\lambda _1}{\lambda _2}|$ meaning convergence is extremely slow if the two eigenvalues are close to each other.

\item[(c)] \textbf{What is the shift and invert approach?}\\
As mentioned previously, if the eigenvalues are close to each other convergence can be extremely slow.\\
The inverse iteration uses the shift and invert tecnique to speed up the convergence.

If the eigenvalues of $A$ are $\lambda_j$, the eigenvalues of $A -\alpha I$ are $\lambda_j - \alpha$, 
and the eigenvalues of $B = (A-\alpha I)^{-1}$ are $\mu_j = \frac{1}{\lambda_j - \alpha}$

If we apply the power method to $B$ we get an improve rate of $|\frac{\mu_2}{\mu_1}| = |\frac{\frac{1}{\lambda_2 - \alpha}}{\frac{1}{\lambda_1 - \alpha}}| = |\frac{\lambda_1 - \alpha}{\lambda_2 - \alpha}|$.\\
It is therefore in our best interests to find an $\alpha$ as close as possible to any eigenvalue.\\
The shift and invert approach is to choose $\alpha$ such that the improve rate is as small as possible.


\item[(d)] \textbf{What is the difference in cost of a single iteration of the power method, compared to the inverse iteration?}\\
Using the power method, each iteration involves a \textit{matrix-vector multiplication}, while the \textit{inverse iteration} requires solving a linear system.
Solving a linear system is way more computationally expensive than a matrix-vector multiplication, having a fast convergence for inverse iteration is necessary in order for it to to be effective; 
this would be impossible when tackling larger issues such as Internet searching.
Inverse iteration must be used for smaller problems or problems with special structure that enables fast direct methods.


\item[(e)] \textbf{What is a Rayleigh quotient and how can it be used for eigenvalue computations?}\\
The Rayleigh quotient is an improvement over the inverse iteration method for finding eigenvalues.\\
It guarantees very fast convergence, in most cases even cubically. To guarantee fast convergence, we change the value of $\alpha$ dynamically to be the Rayleigh quotient in each iteration.\\
Even though Rayleigh quotient iteration is more computationally expensive as it requires to refactor the matrix in each iteration, the cubic convergence makes it worthwhile.

\end{enumerate}

\subsection{Other webgraphs [10 points]}

\subsection{Connectivity matrix and subcliques [5 points]}

\subsection{Connectivity matrix and disjoint subgraphs [10 points]}

\subsection{PageRanks by solving a sparse linear system [40 points]}

\subsection{Quality of the Report [15 points]}


\end{document}

\begin{document}


\setassignment
\setduedate{Wednesday, 12 October 2022, 23:59 AM}

\serieheader{Numerical Computing}{2022}{Student: FULL NAME}{Discussed with: FULL NAME}{Solution for Project 1}{}
\newline

\assignmentpolicy
The purpose of this assignment\footnote{This document is originally based on a SIAM book chapter from \textsl{Numerical Computing with Matlab} from  Clever B. Moler.} is to learn the importance of numerical linear algebra algorithms to solve fundamental  linear algebra problems that occur in search engines.



\section*{PageRank Algorithm }

\subsection{Theory [20 points]}

\begin{enumerate}
\item[(a)] \textbf{What are an eigenvector, an eigenvalue and an eigenbasis?}\\
An eigenvector is a nonzero vector that, when involved in a linear transformation gets stretched and scaled but stays on the same span. \\\\
An eigenvalue, denoted with symbol $\lambda$ represents the factor of how much the eigenvector gets stretched along his direction. \\
The eigenvalue can be negative, as it means the resulting vectors inverts its direction after the linear transformation.  \\\\
An eigenbasis is a diagonal matrix that has eigenvectors as its columns, and eigenvalues along its diagonal.\\
Given a matrix $A$ and a linear transformation $T$, 
if there exists a vector $v$ such that $Tv = \lambda v$,
then $v$ is an eigenvector of $A$ and $\lambda$ is an eigenvalue of A.
If there exists a set of vectors $V$ such that $T(V) = \lambda V$, then $V$ is an eigenbasis of $A$.

% An eigenvector is a vector that doesn't change direction when a linear transformation is applied to it.
% An eigenvalue is a scalar that is multiplied by the eigenvector.
% An eigenbasis is a set of eigenvectors that are linearly independent.

% Given a matrix $A$ and a linear transformation $T$, 
% if there exists a vector $v$ such that $Tv = \lambda v$,
% then $v$ is an eigenvector of $A$ and $\lambda$ is an eigenvalue of A.
% If there exists a set of vectors $V$ such that $T(V) = \lambda V$, then $V$ is an eigenbasis of $A$.


\item[(b)] \textbf{What assumptions should be made to guarantee convergence of the power method?}\\
We need to assume that the eigenvalue to which the power method converges 
is the dominant eigenvalue, and we also need to assume that the randomly-chosen initial vector has a component in the same direction as the eigenvector.\\
Also, to guarantee a faster convergence, $\lambda _1$ and $\lambda _2$ have to be distant, as the asymptotic error constant is $|\frac{\lambda _1}{\lambda _2}|$ meaning convergence is extremely slow if the two eigenvalues are close to each other.

\item[(c)] \textbf{What is the shift and invert approach?}\\
As mentioned previously, if the eigenvalues are close to each other convergence can be extremely slow.\\
The inverse iteration uses the shift and invert tecnique to speed up the convergence.

If the eigenvalues of $A$ are $\lambda_j$, the eigenvalues of $A -\alpha I$ are $\lambda_j - \alpha$, 
and the eigenvalues of $B = (A-\alpha I)^{-1}$ are $\mu_j = \frac{1}{\lambda_j - \alpha}$

If we apply the power method to $B$ we get an improve rate of $|\frac{\mu_2}{\mu_1}| = |\frac{\frac{1}{\lambda_2 - \alpha}}{\frac{1}{\lambda_1 - \alpha}}| = |\frac{\lambda_1 - \alpha}{\lambda_2 - \alpha}|$.\\
It is therefore in our best interests to find an $\alpha$ as close as possible to any eigenvalue.\\
The shift and invert approach is to choose $\alpha$ such that the improve rate is as small as possible.


\item[(d)] \textbf{What is the difference in cost of a single iteration of the power method, compared to the inverse iteration?}\\
Using the power method, each iteration involves a \textit{matrix-vector multiplication}, while the \textit{inverse iteration} requires solving a linear system.
Solving a linear system is way more computationally expensive than a matrix-vector multiplication, having a fast convergence for inverse iteration is necessary in order for it to to be effective; 
this would be impossible when tackling larger issues such as Internet searching.
Inverse iteration must be used for smaller problems or problems with special structure that enables fast direct methods.


\item[(e)] \textbf{What is a Rayleigh quotient and how can it be used for eigenvalue computations?}\\
The Rayleigh quotient is an improvement over the inverse iteration method for finding eigenvalues.\\
It guarantees very fast convergence, in most cases even cubically. To guarantee fast convergence, we change the value of $\alpha$ dynamically to be the Rayleigh quotient in each iteration.\\
Even though Rayleigh quotient iteration is more computationally expensive as it requires to refactor the matrix in each iteration, the cubic convergence makes it worthwhile.

\end{enumerate}

\subsection{Other webgraphs [10 points]}

\subsection{Connectivity matrix and subcliques [5 points]}

\subsection{Connectivity matrix and disjoint subgraphs [10 points]}

\subsection{PageRanks by solving a sparse linear system [40 points]}

\subsection{Quality of the Report [15 points]}


\end{document}

\begin{document}


\setassignment
\setduedate{Wednesday, 12 October 2022, 23:59 AM}

\serieheader{Numerical Computing}{2022}{Student: FULL NAME}{Discussed with: FULL NAME}{Solution for Project 1}{}
\newline

\assignmentpolicy
The purpose of this assignment\footnote{This document is originally based on a SIAM book chapter from \textsl{Numerical Computing with Matlab} from  Clever B. Moler.} is to learn the importance of numerical linear algebra algorithms to solve fundamental  linear algebra problems that occur in search engines.



\section*{PageRank Algorithm }

\subsection{Theory [20 points]}

\begin{enumerate}
\item[(a)] \textbf{What are an eigenvector, an eigenvalue and an eigenbasis?}\\
An eigenvector is a nonzero vector that, when involved in a linear transformation gets stretched and scaled but stays on the same span. \\\\
An eigenvalue, denoted with symbol $\lambda$ represents the factor of how much the eigenvector gets stretched along his direction. \\
The eigenvalue can be negative, as it means the resulting vectors inverts its direction after the linear transformation.  \\\\
An eigenbasis is a diagonal matrix that has eigenvectors as its columns, and eigenvalues along its diagonal.\\
Given a matrix $A$ and a linear transformation $T$, 
if there exists a vector $v$ such that $Tv = \lambda v$,
then $v$ is an eigenvector of $A$ and $\lambda$ is an eigenvalue of A.
If there exists a set of vectors $V$ such that $T(V) = \lambda V$, then $V$ is an eigenbasis of $A$.

% An eigenvector is a vector that doesn't change direction when a linear transformation is applied to it.
% An eigenvalue is a scalar that is multiplied by the eigenvector.
% An eigenbasis is a set of eigenvectors that are linearly independent.

% Given a matrix $A$ and a linear transformation $T$, 
% if there exists a vector $v$ such that $Tv = \lambda v$,
% then $v$ is an eigenvector of $A$ and $\lambda$ is an eigenvalue of A.
% If there exists a set of vectors $V$ such that $T(V) = \lambda V$, then $V$ is an eigenbasis of $A$.


\item[(b)] \textbf{What assumptions should be made to guarantee convergence of the power method?}\\
We need to assume that the eigenvalue to which the power method converges 
is the dominant eigenvalue, and we also need to assume that the randomly-chosen initial vector has a component in the same direction as the eigenvector.\\
Also, to guarantee a faster convergence, $\lambda _1$ and $\lambda _2$ have to be distant, as the asymptotic error constant is $|\frac{\lambda _1}{\lambda _2}|$ meaning convergence is extremely slow if the two eigenvalues are close to each other.

\item[(c)] \textbf{What is the shift and invert approach?}\\
As mentioned previously, if the eigenvalues are close to each other convergence can be extremely slow.\\
The inverse iteration uses the shift and invert tecnique to speed up the convergence.

If the eigenvalues of $A$ are $\lambda_j$, the eigenvalues of $A -\alpha I$ are $\lambda_j - \alpha$, 
and the eigenvalues of $B = (A-\alpha I)^{-1}$ are $\mu_j = \frac{1}{\lambda_j - \alpha}$

If we apply the power method to $B$ we get an improve rate of $|\frac{\mu_2}{\mu_1}| = |\frac{\frac{1}{\lambda_2 - \alpha}}{\frac{1}{\lambda_1 - \alpha}}| = |\frac{\lambda_1 - \alpha}{\lambda_2 - \alpha}|$.\\
It is therefore in our best interests to find an $\alpha$ as close as possible to any eigenvalue.\\
The shift and invert approach is to choose $\alpha$ such that the improve rate is as small as possible.


\item[(d)] \textbf{What is the difference in cost of a single iteration of the power method, compared to the inverse iteration?}\\
Using the power method, each iteration involves a \textit{matrix-vector multiplication}, while the \textit{inverse iteration} requires solving a linear system.
Solving a linear system is way more computationally expensive than a matrix-vector multiplication, having a fast convergence for inverse iteration is necessary in order for it to to be effective; 
this would be impossible when tackling larger issues such as Internet searching.
Inverse iteration must be used for smaller problems or problems with special structure that enables fast direct methods.


\item[(e)] \textbf{What is a Rayleigh quotient and how can it be used for eigenvalue computations?}\\
The Rayleigh quotient is an improvement over the inverse iteration method for finding eigenvalues.\\
It guarantees very fast convergence, in most cases even cubically. To guarantee fast convergence, we change the value of $\alpha$ dynamically to be the Rayleigh quotient in each iteration.\\
Even though Rayleigh quotient iteration is more computationally expensive as it requires to refactor the matrix in each iteration, the cubic convergence makes it worthwhile.

\end{enumerate}

\subsection{Other webgraphs [10 points]}

\subsection{Connectivity matrix and subcliques [5 points]}

\subsection{Connectivity matrix and disjoint subgraphs [10 points]}

\subsection{PageRanks by solving a sparse linear system [40 points]}

\subsection{Quality of the Report [15 points]}


\end{document}



\hyphenation{PageRank}
\hyphenation{PageRanks}


\begin{document}


\setassignment
\setduedate{Wednesday, 26 October 2022, 23:59 AM}

\serieheader{Numerical Computing}{2022}{Student: FULL NAME}{Discussed with: FULL NAME}{Solution for Project 2}{}
\newline

\assignmentpolicy


The purpose of this assignment\footnote{This document is originally 
based on a blog from Cleve Moler, who wrote a fantastic blog post about the Lake Arrowhead graph, and John 
Gilbert, who initially created the coauthor graph from the 1993 Householder Meeting. You can find more information 
at \url{http://blogs.mathworks.com/cleve/2013/06/10/lake-arrowhead-coauthor-graph/}.  Most of this assignment is derived
from this archived work.} is to learn the importance of sparse linear algebra algorithms to solve fundamental 
questions in social network analyses.
We will use the coauthor graph from the Householder Meeting and the social network of friendships from Zachary's karate club~\cite{karate}.
These two graphs are one of the first examples where matrix methods were used in computational social network analyses.

\section*{Social Networks [Total: 85 points + 15 points for report quality]}

\subsection{The Reverse Cuthill McKee Ordering [10 points]}

\subsection{Sparse Matrix Factorization [20 points]}

\subsubsection*{a)}

\[
    A_{ij} = 
    \begin{cases}
        1 & \text{if $i=1$ or $j=1$  or $j=n$ and $i \neq j$} \\
        n + i -1 & \text{if $i=j$}\\
        0 & \text{otherwise}
    \end{cases}
\]

% construct matrix:
% - A_{ij} = 1 if i=1 or j=1 or j=n and i != j
% - A_{ij} = n + i -1 if i=j
% - A_{ij} = 0 otherwise

\[A = 
\begin{bmatrix}
    1 & 1 & 1 & 1 & 1 & 1 & 1 & 1 & 1 & 1 \\
    1 & 11 & 0 & 0 & 0 & 0 & 0 & 0 & 0 & 1 \\
    1 & 0 & 12 & 0 & 0 & 0 & 0 & 0 & 0 & 1 \\
    1 & 0 & 0 & 13 & 0 & 0 & 0 & 0 & 0 & 1 \\
    1 & 0 & 0 & 0 & 14 & 0 & 0 & 0 & 0 & 1 \\
    1 & 0 & 0 & 0 & 0 & 15 & 0 & 0 & 0 & 1 \\
    1 & 0 & 0 & 0 & 0 & 0 & 16 & 0 & 0 & 1 \\
    1 & 0 & 0 & 0 & 0 & 0 & 0 & 17 & 0 & 1 \\
    1 & 0 & 0 & 0 & 0 & 0 & 0 & 0 & 18 & 1\\
    1 & 1 & 1 & 1 & 1 & 1 & 1 & 1 & 1 & 1
\end{bmatrix}
\]

Number of non-zero elements: 44;

\subsubsection*{b)}

A matrix $A \in R^{n \times n}$ will have $n \times n$ entries.
The non-zero entries are:
\begin{itemize}
    \item $n$ entries in the first row
    \item $n$ entries in the last row
    \item $n$ entries in the first column
    \item $n$ entries in the last column
    \item $n$ entries in the diagonal
\end{itemize}
So $5n$ minus the entries that are counted twice or more:
\begin{itemize}
    \item $A_{11}$ is counted in the first row, the first column and in the diagonal
    \item $A_{nn}$ is counted in the last row, the last column and in the diagonal
    \item $A_{1n}$ is counted in the first row and the last column
    \item $A_{n1}$ is counted in the last row and the first column
\end{itemize}
So the total number of non-zero entries is: $5n - 2 -2 -1-1 = 5n-6$.

\subsubsection*{d)}
% TODO: Julia code does not run

\subsubsection*{e)}



\subsection{Degree Centrality [5 points]}

\subsection{The Connectivity of the Coauthors [5 points]}

\subsection{PageRank of the Coauthor Graph [5 points]}

\subsection{Zachary's karate club: social network of friendships between 34 members [40 points]}

\begin{thebibliography}{99}
\bibitem{karate} The social network of a karate club at a US university, M.~E.~J. Newman and M. Girvan, Phys. Rev. E 69,026113 (2004)
pp. 219-229.
\end{thebibliography}


\end{document}
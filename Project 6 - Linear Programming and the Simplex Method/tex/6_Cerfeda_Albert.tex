\documentclass[unicode,11pt,a4paper,oneside,numbers=endperiod,openany]{scrartcl}

\documentclass[unicode,11pt,a4paper,oneside,numbers=endperiod,openany]{scrartcl}

\renewcommand{\thesubsection}{\arabic{subsection}}

\documentclass[unicode,11pt,a4paper,oneside,numbers=endperiod,openany]{scrartcl}

\renewcommand{\thesubsection}{\arabic{subsection}}

\documentclass[unicode,11pt,a4paper,oneside,numbers=endperiod,openany]{scrartcl}

\renewcommand{\thesubsection}{\arabic{subsection}}

\input{assignment.sty}
\begin{document}


\setassignment
\setduedate{Wednesday, 12 October 2022, 23:59 AM}

\serieheader{Numerical Computing}{2022}{Student: FULL NAME}{Discussed with: FULL NAME}{Solution for Project 1}{}
\newline

\assignmentpolicy
The purpose of this assignment\footnote{This document is originally based on a SIAM book chapter from \textsl{Numerical Computing with Matlab} from  Clever B. Moler.} is to learn the importance of numerical linear algebra algorithms to solve fundamental  linear algebra problems that occur in search engines.



\section*{PageRank Algorithm }

\subsection{Theory [20 points]}

\begin{enumerate}
\item[(a)] \textbf{What are an eigenvector, an eigenvalue and an eigenbasis?}\\
An eigenvector is a nonzero vector that, when involved in a linear transformation gets stretched and scaled but stays on the same span. \\\\
An eigenvalue, denoted with symbol $\lambda$ represents the factor of how much the eigenvector gets stretched along his direction. \\
The eigenvalue can be negative, as it means the resulting vectors inverts its direction after the linear transformation.  \\\\
An eigenbasis is a diagonal matrix that has eigenvectors as its columns, and eigenvalues along its diagonal.\\
Given a matrix $A$ and a linear transformation $T$, 
if there exists a vector $v$ such that $Tv = \lambda v$,
then $v$ is an eigenvector of $A$ and $\lambda$ is an eigenvalue of A.
If there exists a set of vectors $V$ such that $T(V) = \lambda V$, then $V$ is an eigenbasis of $A$.

% An eigenvector is a vector that doesn't change direction when a linear transformation is applied to it.
% An eigenvalue is a scalar that is multiplied by the eigenvector.
% An eigenbasis is a set of eigenvectors that are linearly independent.

% Given a matrix $A$ and a linear transformation $T$, 
% if there exists a vector $v$ such that $Tv = \lambda v$,
% then $v$ is an eigenvector of $A$ and $\lambda$ is an eigenvalue of A.
% If there exists a set of vectors $V$ such that $T(V) = \lambda V$, then $V$ is an eigenbasis of $A$.


\item[(b)] \textbf{What assumptions should be made to guarantee convergence of the power method?}\\
We need to assume that the eigenvalue to which the power method converges 
is the dominant eigenvalue, and we also need to assume that the randomly-chosen initial vector has a component in the same direction as the eigenvector.\\
Also, to guarantee a faster convergence, $\lambda _1$ and $\lambda _2$ have to be distant, as the asymptotic error constant is $|\frac{\lambda _1}{\lambda _2}|$ meaning convergence is extremely slow if the two eigenvalues are close to each other.

\item[(c)] \textbf{What is the shift and invert approach?}\\
As mentioned previously, if the eigenvalues are close to each other convergence can be extremely slow.\\
The inverse iteration uses the shift and invert tecnique to speed up the convergence.

If the eigenvalues of $A$ are $\lambda_j$, the eigenvalues of $A -\alpha I$ are $\lambda_j - \alpha$, 
and the eigenvalues of $B = (A-\alpha I)^{-1}$ are $\mu_j = \frac{1}{\lambda_j - \alpha}$

If we apply the power method to $B$ we get an improve rate of $|\frac{\mu_2}{\mu_1}| = |\frac{\frac{1}{\lambda_2 - \alpha}}{\frac{1}{\lambda_1 - \alpha}}| = |\frac{\lambda_1 - \alpha}{\lambda_2 - \alpha}|$.\\
It is therefore in our best interests to find an $\alpha$ as close as possible to any eigenvalue.\\
The shift and invert approach is to choose $\alpha$ such that the improve rate is as small as possible.


\item[(d)] \textbf{What is the difference in cost of a single iteration of the power method, compared to the inverse iteration?}\\
Using the power method, each iteration involves a \textit{matrix-vector multiplication}, while the \textit{inverse iteration} requires solving a linear system.
Solving a linear system is way more computationally expensive than a matrix-vector multiplication, having a fast convergence for inverse iteration is necessary in order for it to to be effective; 
this would be impossible when tackling larger issues such as Internet searching.
Inverse iteration must be used for smaller problems or problems with special structure that enables fast direct methods.


\item[(e)] \textbf{What is a Rayleigh quotient and how can it be used for eigenvalue computations?}\\
The Rayleigh quotient is an improvement over the inverse iteration method for finding eigenvalues.\\
It guarantees very fast convergence, in most cases even cubically. To guarantee fast convergence, we change the value of $\alpha$ dynamically to be the Rayleigh quotient in each iteration.\\
Even though Rayleigh quotient iteration is more computationally expensive as it requires to refactor the matrix in each iteration, the cubic convergence makes it worthwhile.

\end{enumerate}

\subsection{Other webgraphs [10 points]}

\subsection{Connectivity matrix and subcliques [5 points]}

\subsection{Connectivity matrix and disjoint subgraphs [10 points]}

\subsection{PageRanks by solving a sparse linear system [40 points]}

\subsection{Quality of the Report [15 points]}


\end{document}

\begin{document}


\setassignment
\setduedate{Wednesday, 12 October 2022, 23:59 AM}

\serieheader{Numerical Computing}{2022}{Student: FULL NAME}{Discussed with: FULL NAME}{Solution for Project 1}{}
\newline

\assignmentpolicy
The purpose of this assignment\footnote{This document is originally based on a SIAM book chapter from \textsl{Numerical Computing with Matlab} from  Clever B. Moler.} is to learn the importance of numerical linear algebra algorithms to solve fundamental  linear algebra problems that occur in search engines.



\section*{PageRank Algorithm }

\subsection{Theory [20 points]}

\begin{enumerate}
\item[(a)] \textbf{What are an eigenvector, an eigenvalue and an eigenbasis?}\\
An eigenvector is a nonzero vector that, when involved in a linear transformation gets stretched and scaled but stays on the same span. \\\\
An eigenvalue, denoted with symbol $\lambda$ represents the factor of how much the eigenvector gets stretched along his direction. \\
The eigenvalue can be negative, as it means the resulting vectors inverts its direction after the linear transformation.  \\\\
An eigenbasis is a diagonal matrix that has eigenvectors as its columns, and eigenvalues along its diagonal.\\
Given a matrix $A$ and a linear transformation $T$, 
if there exists a vector $v$ such that $Tv = \lambda v$,
then $v$ is an eigenvector of $A$ and $\lambda$ is an eigenvalue of A.
If there exists a set of vectors $V$ such that $T(V) = \lambda V$, then $V$ is an eigenbasis of $A$.

% An eigenvector is a vector that doesn't change direction when a linear transformation is applied to it.
% An eigenvalue is a scalar that is multiplied by the eigenvector.
% An eigenbasis is a set of eigenvectors that are linearly independent.

% Given a matrix $A$ and a linear transformation $T$, 
% if there exists a vector $v$ such that $Tv = \lambda v$,
% then $v$ is an eigenvector of $A$ and $\lambda$ is an eigenvalue of A.
% If there exists a set of vectors $V$ such that $T(V) = \lambda V$, then $V$ is an eigenbasis of $A$.


\item[(b)] \textbf{What assumptions should be made to guarantee convergence of the power method?}\\
We need to assume that the eigenvalue to which the power method converges 
is the dominant eigenvalue, and we also need to assume that the randomly-chosen initial vector has a component in the same direction as the eigenvector.\\
Also, to guarantee a faster convergence, $\lambda _1$ and $\lambda _2$ have to be distant, as the asymptotic error constant is $|\frac{\lambda _1}{\lambda _2}|$ meaning convergence is extremely slow if the two eigenvalues are close to each other.

\item[(c)] \textbf{What is the shift and invert approach?}\\
As mentioned previously, if the eigenvalues are close to each other convergence can be extremely slow.\\
The inverse iteration uses the shift and invert tecnique to speed up the convergence.

If the eigenvalues of $A$ are $\lambda_j$, the eigenvalues of $A -\alpha I$ are $\lambda_j - \alpha$, 
and the eigenvalues of $B = (A-\alpha I)^{-1}$ are $\mu_j = \frac{1}{\lambda_j - \alpha}$

If we apply the power method to $B$ we get an improve rate of $|\frac{\mu_2}{\mu_1}| = |\frac{\frac{1}{\lambda_2 - \alpha}}{\frac{1}{\lambda_1 - \alpha}}| = |\frac{\lambda_1 - \alpha}{\lambda_2 - \alpha}|$.\\
It is therefore in our best interests to find an $\alpha$ as close as possible to any eigenvalue.\\
The shift and invert approach is to choose $\alpha$ such that the improve rate is as small as possible.


\item[(d)] \textbf{What is the difference in cost of a single iteration of the power method, compared to the inverse iteration?}\\
Using the power method, each iteration involves a \textit{matrix-vector multiplication}, while the \textit{inverse iteration} requires solving a linear system.
Solving a linear system is way more computationally expensive than a matrix-vector multiplication, having a fast convergence for inverse iteration is necessary in order for it to to be effective; 
this would be impossible when tackling larger issues such as Internet searching.
Inverse iteration must be used for smaller problems or problems with special structure that enables fast direct methods.


\item[(e)] \textbf{What is a Rayleigh quotient and how can it be used for eigenvalue computations?}\\
The Rayleigh quotient is an improvement over the inverse iteration method for finding eigenvalues.\\
It guarantees very fast convergence, in most cases even cubically. To guarantee fast convergence, we change the value of $\alpha$ dynamically to be the Rayleigh quotient in each iteration.\\
Even though Rayleigh quotient iteration is more computationally expensive as it requires to refactor the matrix in each iteration, the cubic convergence makes it worthwhile.

\end{enumerate}

\subsection{Other webgraphs [10 points]}

\subsection{Connectivity matrix and subcliques [5 points]}

\subsection{Connectivity matrix and disjoint subgraphs [10 points]}

\subsection{PageRanks by solving a sparse linear system [40 points]}

\subsection{Quality of the Report [15 points]}


\end{document}

\begin{document}


\setassignment
\setduedate{Wednesday, 12 October 2022, 23:59 AM}

\serieheader{Numerical Computing}{2022}{Student: FULL NAME}{Discussed with: FULL NAME}{Solution for Project 1}{}
\newline

\assignmentpolicy
The purpose of this assignment\footnote{This document is originally based on a SIAM book chapter from \textsl{Numerical Computing with Matlab} from  Clever B. Moler.} is to learn the importance of numerical linear algebra algorithms to solve fundamental  linear algebra problems that occur in search engines.



\section*{PageRank Algorithm }

\subsection{Theory [20 points]}

\begin{enumerate}
\item[(a)] \textbf{What are an eigenvector, an eigenvalue and an eigenbasis?}\\
An eigenvector is a nonzero vector that, when involved in a linear transformation gets stretched and scaled but stays on the same span. \\\\
An eigenvalue, denoted with symbol $\lambda$ represents the factor of how much the eigenvector gets stretched along his direction. \\
The eigenvalue can be negative, as it means the resulting vectors inverts its direction after the linear transformation.  \\\\
An eigenbasis is a diagonal matrix that has eigenvectors as its columns, and eigenvalues along its diagonal.\\
Given a matrix $A$ and a linear transformation $T$, 
if there exists a vector $v$ such that $Tv = \lambda v$,
then $v$ is an eigenvector of $A$ and $\lambda$ is an eigenvalue of A.
If there exists a set of vectors $V$ such that $T(V) = \lambda V$, then $V$ is an eigenbasis of $A$.

% An eigenvector is a vector that doesn't change direction when a linear transformation is applied to it.
% An eigenvalue is a scalar that is multiplied by the eigenvector.
% An eigenbasis is a set of eigenvectors that are linearly independent.

% Given a matrix $A$ and a linear transformation $T$, 
% if there exists a vector $v$ such that $Tv = \lambda v$,
% then $v$ is an eigenvector of $A$ and $\lambda$ is an eigenvalue of A.
% If there exists a set of vectors $V$ such that $T(V) = \lambda V$, then $V$ is an eigenbasis of $A$.


\item[(b)] \textbf{What assumptions should be made to guarantee convergence of the power method?}\\
We need to assume that the eigenvalue to which the power method converges 
is the dominant eigenvalue, and we also need to assume that the randomly-chosen initial vector has a component in the same direction as the eigenvector.\\
Also, to guarantee a faster convergence, $\lambda _1$ and $\lambda _2$ have to be distant, as the asymptotic error constant is $|\frac{\lambda _1}{\lambda _2}|$ meaning convergence is extremely slow if the two eigenvalues are close to each other.

\item[(c)] \textbf{What is the shift and invert approach?}\\
As mentioned previously, if the eigenvalues are close to each other convergence can be extremely slow.\\
The inverse iteration uses the shift and invert tecnique to speed up the convergence.

If the eigenvalues of $A$ are $\lambda_j$, the eigenvalues of $A -\alpha I$ are $\lambda_j - \alpha$, 
and the eigenvalues of $B = (A-\alpha I)^{-1}$ are $\mu_j = \frac{1}{\lambda_j - \alpha}$

If we apply the power method to $B$ we get an improve rate of $|\frac{\mu_2}{\mu_1}| = |\frac{\frac{1}{\lambda_2 - \alpha}}{\frac{1}{\lambda_1 - \alpha}}| = |\frac{\lambda_1 - \alpha}{\lambda_2 - \alpha}|$.\\
It is therefore in our best interests to find an $\alpha$ as close as possible to any eigenvalue.\\
The shift and invert approach is to choose $\alpha$ such that the improve rate is as small as possible.


\item[(d)] \textbf{What is the difference in cost of a single iteration of the power method, compared to the inverse iteration?}\\
Using the power method, each iteration involves a \textit{matrix-vector multiplication}, while the \textit{inverse iteration} requires solving a linear system.
Solving a linear system is way more computationally expensive than a matrix-vector multiplication, having a fast convergence for inverse iteration is necessary in order for it to to be effective; 
this would be impossible when tackling larger issues such as Internet searching.
Inverse iteration must be used for smaller problems or problems with special structure that enables fast direct methods.


\item[(e)] \textbf{What is a Rayleigh quotient and how can it be used for eigenvalue computations?}\\
The Rayleigh quotient is an improvement over the inverse iteration method for finding eigenvalues.\\
It guarantees very fast convergence, in most cases even cubically. To guarantee fast convergence, we change the value of $\alpha$ dynamically to be the Rayleigh quotient in each iteration.\\
Even though Rayleigh quotient iteration is more computationally expensive as it requires to refactor the matrix in each iteration, the cubic convergence makes it worthwhile.

\end{enumerate}

\subsection{Other webgraphs [10 points]}

\subsection{Connectivity matrix and subcliques [5 points]}

\subsection{Connectivity matrix and disjoint subgraphs [10 points]}

\subsection{PageRanks by solving a sparse linear system [40 points]}

\subsection{Quality of the Report [15 points]}


\end{document}


\usepackage{pgfplots} 
\usetikzlibrary{automata,topaths}

\makeatletter
\newcommand{\pgfplotsdrawaxis}{\pgfplots@draw@axis}
\makeatother
\pgfplotsset{only axis on top/.style={axis on top=false, after end axis/.code={
             \pgfplotsset{axis line style=opaque, ticklabel style=opaque, tick style=opaque,
                          grid=none}\pgfplotsdrawaxis}}}

\newcommand{\drawge}{-- (rel axis cs:1,0) -- (rel axis cs:1,1) -- (rel axis cs:0,1) \closedcycle}
\newcommand{\drawle}{-- (rel axis cs:1,1) -- (rel axis cs:1,0) -- (rel axis cs:0,0) \closedcycle}
\newenvironment{rcases}
  {\left.\begin{aligned}}
  {\end{aligned}\right\rbrace}


\begin{document}


\setassignment
\setduedate{Wednesday, December 21, 2022, 11:59 PM}

\serieheader{Numerical Computing}{2022}{Student: Albert Cerfeda}{}{Solution for Project 6}{}
\newline

\assignmentpolicy


The purpose of this project is to implement the Simplex Method to find the solution of linear programs, involving both the minimisation and the maximisation of the objective function.
\tableofcontents
\clearpage

\section{Graphical Solution of Linear Programming Problems [20 points]}
\subsection*{1.1 Minimization problem}
\begin{equation*}
    \begin{split}
    \text{min  } z & = 4x+y \\
     \text{\textbf{s.t}  } x+2y&\leq 40\\
     x+y&\geq 30 \\
     2x+3y&\geq 72 \\
     x,y &\geq 0
    \end{split}
\end{equation*}
Let us plot the inequalities on the Cartesian plane:\\
\begin{figure}[htpb] 
    \centering 
    \begin{tikzpicture} 
      \begin{axis}[only axis on top,
        axis line style=very thick, 
        axis x line=bottom, 
        axis y line=left, 
         ymin=0,ymax=40,xmin=0,xmax=40, 
         xlabel=$x$, ylabel=$y$,grid=major 
      ] 
        
        \addplot [draw=none, fill=red, fill opacity=0.25, domain=-10:40]
        {(40-x)/2} \drawle; 
        \addplot [draw=none, fill=red, fill opacity=0.25, domain=-10:40]
        {30-x} \drawge;  
        \addplot [draw=none, fill=red, fill opacity=0.25, domain=-10:40]
        {(72-(2*x))/3} \drawge;
        \addplot [draw=none, fill=red, fill opacity=0.20, domain=-10:40]
        {0} \drawge;
        
        \addplot[very thick, domain=-10:40, ] {(40-x)/2} ; 
        \addplot[very thick, domain=-10:40, ] {30-x} ; 
        \addplot[very thick, domain=-10:40, ] {(72-(2*x))/3} ; 

        \addplot[mark=*,only marks] coordinates {(24,8)} node[above, xshift=0.7cm] {$P_1(24,8)$};
        \addplot[mark=*,only marks] coordinates {(36,0)} node[above, xshift=-1cm] {$P_2(36,0)$};
        \addplot[mark=*,only marks] coordinates {(40,0)} node[above, xshift=-0.7cm, yshift=0.5cm] {$P_3(40,0)$};
        

      \end{axis}
    \end{tikzpicture}
    \caption{Feasibile region for the Minimization problem} 
  \end{figure} 

We notice three intersection vertices:\\
\begin{tabular}{ ccc } 
    $x=24$ & $x=36,y=0$ & $x=40,y=0$\\
\end{tabular}\\
The point with $x=24$ is the intersection of inequalities $x+2y\leq 40$ and $2x+3y \geq 72$. Let's find the $y$ component of the intersection point:\\
$24+2y=40 \Rightarrow y=8$.\\\\
We now need to calculate the objective function value for each vertex:\\
\begin{tabular}{ ccc } 
    $p_1 = (24,8)$ & $p_2 = (36,0)$ & $p3=(40,0)$\\
\end{tabular}\\
\begin{equation*}
    \begin{split}
     z_1 &= 4(24)+ 8 = 104 \\
     z_2 &= 4(36) + 10 = 144 \\
     z_3 &= 4(40) + 0 = 160 \\
    \end{split}
\end{equation*}
Therefore $\text{min  } z = z_1 = 104$

\subsection*{1.2 Tailor maximisation problem}
The tailor in our problem sells two types of trousers, which we will identify as $x$ for the first type of trousers and $y$ for the second type of trousers.\\
The manufacturing cost of trouser $x$ equals $m_x = 25$ and for the case of trouser $y$ equals $m_y = 40$.\\
The retail price for trouser $x$ equals $85$  and for trouser $y$ equals $110$.\\
We denote the net profit for each trouser as $n_T$ where $T$ is the type of trouser. We can infer that $n_x=85-25=60$ and $n_y=110-40=70$.\\\\
\textbf{Note:} in the text of the exercise it is written \textit{The tailor estimates a total monthly demand of 265 trousers}. I interpreted it such that the tailor does not expect more than 365 trousers, i.e $x+y\leq265$. Were it to be intepreted as if the tailor was to expect at least 265 trousers, it would been have modeled with $x+y\geq265$.\\
Let us model the objective function aswell as the constraints:\\

% TODO: Plot constraints
\begin{equation*}
    \begin{split}
    \text{max  } z & = 60x+70y \\
     \text{\textbf{s.t}  } x+y&\leq 265\\
     25x+40y&\leq 7000 \\
     x,y &\geq 0, z\geq 0
    \end{split}
\end{equation*}\\
That is, the tailor wants to \textit{maximize} the net profit from selling both types of trousers, expects to sell not more than $265$ trousers, and spend less that $7000$ in raw materials.\\
The non-negativity constraints are intuitive as it is not possible to sell a negative amount of trousers.\\
Let's solve the systems of inequalities:\\
\begin{align*}
    & \begin{cases}
        y=0\\
        x+y=265
    \end{cases}\\
    &\begin{cases}
        y=0\\
        x=265
    \end{cases}\\
    &\Rightarrow P_1(265, 0) \\\\
    & \begin{cases}
        x+y=265\\
        25x+40y=7000
    \end{cases}\\
    &\begin{cases}
        x=265-y\\
        6625-25y+40y=7000
    \end{cases}\\
    &\begin{cases}
        x=265-y\\
        y=25
    \end{cases}\\
    &\Rightarrow P_2(240, 25)\\\\
    & \begin{cases}
        x=0\\
        40y=7000
    \end{cases}\\
    &\Rightarrow P_3(0, 175)
\end{align*}\\
Let's plot the inequalities on the Cartesian Plane:
\clearpage
\begin{figure}[h!] 
    \centering 
    \begin{tikzpicture} 
      \begin{axis}[only axis on top,
        axis line style=very thick, 
        axis x line=bottom, 
        axis y line=left, 
         ymin=0,ymax=285,xmin=0,xmax=285, 
         xlabel=$x$, ylabel=$y$,grid=major 
      ] 
        
        \addplot [draw=none, fill=red, fill opacity=0.25, domain=-10:285]
        {265-x} \drawle; 
        \addplot [draw=none, fill=red, fill opacity=0.25, domain=-10:285]
        {(5*(-x+280))/8} \drawle;
        \addplot [draw=none, fill=red, fill opacity=0.20, domain=-10:285]
        {0} \drawge;
        
        \addplot[very thick, domain=-10:285, ] {265-x} ; 
        \addplot[very thick, domain=-10:285, ] {(5*(-x+280))/8} ; 

        \addplot[mark=*,only marks] coordinates {(265,0)} node[above, yshift=-0.1cm, xshift=-1.1cm] {$P_1(265,0)$};
        \addplot[mark=*,only marks] coordinates {(240,25)} node[above, yshift=0.4cm, xshift=0.2cm] {$P_2(240,25)$};
        \addplot[mark=*,only marks] coordinates {(0,175)} node[above, xshift=0.8cm, yshift=0cm] {$P_3(0,175)$};
        

      \end{axis} 
    \end{tikzpicture} 
    \caption{Feasible region for the Tailor maximisation problem} 
  \end{figure} 

\section{Implementation of the Simplex Method [30 points]}
The code has been successfully implemented in Julia.\\
When running the provided tests, all of them pass without incurring in the maximum iterations upper bound.\\
Please checkout the source files in the \verb|src/| folder, as it contains my implementation.

\section{Applications to Real-Life Example: Cargo Aircraft [25 points]}
\subsection*{3.1 Modelling the linear program}
Before modelling the linear problem we need to define the variable $t$:\\
$$
t_{xn} \text{where $x$ is the tonnes allocated from cargo $C_x$ and $n$ is the target Storage compartment $S_n$}
$$
There $t_14$ are the tonnes from Cargo 1 ie $C_1$ put into Storage compartment 4 ie $S_r$.
\begin{align*}
  \text{max} z &= (135*1*t_{11}) + (135*1.1*t_{12}) \\
  & +(200*1*t_{21})+(200*1.1*t_{22})\\
  & +(200*1.2*t_{23})+(200*1.3*t_{24})\\
  & +(410*1*t_{31})+(410*1.1*t_{32})\\
  & +(410*1.2*t_{33})+(410*1.3*t_{34})\\
  & +(520*1*t_{41})+(520*1.1*t_{42})\\
  & +(520*1.2*t_{43})+(520*1.3*t_{44})
\end{align*}
The objective function expresses the need to maximize the amount of profit from properly splitting the cargo in the available Storage compartments from the various Cargo containers.
\begin{align*}
  \text{\textbf{s.t. }} & \text{\textbf{Compartment weight capacity constraints}} \\
  & t_{11} + t_{21} + t_{31} + t_{45} \leq 18\\
  & t_{12} + t_{22} + t_{32} + t_{46} \leq 32\\
  & t_{13} + t_{23} + t_{33} + t_{47} \leq 25\\
  & t_{14} + t_{24} + t_{34} + t_{48} \leq 17\\
\end{align*}
Here we model the constraints coming from all the cargo loaded into the Storage compartment, not having to exceed the weight limit of the Storage compartment itself.
\begin{align*}
   & \text{\textbf{Compartment volume capacity constraints}} \\
  & 320*t_{11} + 510*t_{21} + 630*t_{31} + 125*t_{45} \leq 11930\\
  & 320*t_{12} + 510*t_{22} + 630*t_{32} + 125*t_{46} \leq 22552\\
  & 320*t_{13} + 510*t_{23} + 630*t_{33} + 125*t_{47} \leq 11209\\
  & 320*t_{14} + 510*t_{24} + 630*t_{34} + 125*t_{48} \leq 5870\\
\end{align*}
Here we model the constraints coming from all the cargo loaded into the Storage compartment, not having to exceed the volume limit of the Storage compartment itself.
\begin{align*}
  & \text{\textbf{Cargo volume availability constraints}} \\
  & (t_{11}+t_{12}+t_{13}+t_{14}) \leq 16\\
  & (t_{21}+t_{22}+t_{23}+t_{24}) \leq 32\\
  & (t_{31}+t_{32}+t_{33}+t_{34}) \leq 40\\
  & (t_{41}+t_{42}+t_{43}+t_{44}) \leq 28\\
\end{align*}
Here we model the maximum amount of cargo inside each container. Thus, we can't split more cargo than there actually is.
\begin{align*}
  & \text{\textbf{Non-negativity constraints}} \\
  & (t_{11}+t_{12}+t_{13}+t_{14}) \geq 0\\
  & (t_{21}+t_{22}+t_{23}+t_{24}) \geq 0\\
  & (t_{31}+t_{32}+t_{33}+t_{34}) \geq 0\\
  & (t_{41}+t_{42}+t_{43}+t_{44}) \geq 0\\
\end{align*}
Obviously the non-negativity constraints come from the impossibility of allocating a negative cargo weight.

\subsection{Solving the linear problem using Julia}
Let us model the problem in Julia for solving it:\\
% \begin{figure}[h!]
%   \begin{minted}[
%   frame=lines,
%   framesep=2mm,
%   linenos
%   ]{julia}
% cd(dirname(@__FILE__))
% include("simplex.jl")
% include("simplexSolve.jl")
% include("standardize.jl")
% include("auxiliary.jl")
% include("printSol.jl")

% function strcmp(str1, str2)
%     return cmp(str1, str2) == 0
% end

% # Input arguments:
% #   type = "max' for maximization; 'min" for minimization
% #   A    = matrix holding the constraints coefficients
% #   h    = coefficients of the constraints inequalities [rhs vector]
% #   c    = coefficients of the objective functions
% #   sign = vector holding information about the constraints if the system()
% #          needs to be standardized [-1: less or equal, 0: equal, 1:vgreater | equal]

% # Coefficient matrix A omitted
% # A ...

% type = "max"
% h = [ 18 32 25 17 11930 22552 11209 5870 16 32 40 28 ]'
% sign = [ -1 -1 -1 -1  -1 -1 -1 -1  -1 -1 -1 -1 ]
% c = [ 135 135*1.1 135*1.2 135*1.3    200 200*1.1 200*1.2 200*1.3    410 410*1.1 410*1.2 410*1.3    520 520*1.1 520*1.2 520*1.3]  
% z, x_B, index_B = simplex(type, A, h, c, sign)

% @show z, x_B, index_B
% \end{minted}
% \caption{Julia code for solving the real world linear problem for the Cargo Aircraft }
% \end{figure}

\textbf{Important}: Coefficient matrix A has been omitted from the report as it wouldnt have fit the page. Please check it out inside file \verb|src/exercise2.jl|.
The type has been set to "max" as this is a maximisation problem.
Matrix $A$ holds all the coefficients for the constraints, as modelled in the subsection before.
Vector $h$ contains all the right-hand sides of the constrain inequalities.
Vector $c$ contains all the coefficients of the objective function.
Vector $sign$ contains the information on whether each inequality sign has to be flipped. As this is a maximisation problem, and every explicit constraints inequality in our constraint is $\leq$, we have to flip all of them. 






\section{Cycling and Degeneracy [10 points]}



\end{document}
